\documentclass{article}
\usepackage{syntax}
\usepackage[utf8]{inputenc}
\usepackage{amsmath}
\usepackage{amssymb}
\usepackage{bussproofs}
\usepackage{todonotes}
\usepackage{xspace}
\usepackage{tikz}
\usepackage{graphicx}
\usepackage[landscape]{geometry}

\title{Opal Semantics: Technical Report}
\author{Alex Renda, Harrison Goldstein, Sarah Bird, Chris Quirk, Adrian Sampson}
\date{\today}

\newcommand{\skipcom}{\texttt{skip}}
\newcommand{\seqcom}[2]{#1;\ #2}
\newcommand{\ifcom}[3]{\texttt{if}\ #1\ \texttt{then}\ \{#2\}\ \texttt{else}\ \{#3\}}
\newcommand{\withcom}[2]{\texttt{with}\ #1\ \texttt{do}\ \{#2\}}
\newcommand{\atcom}[2]{\texttt{at}\ #1\ \texttt{do}\ \{#2\}}
\newcommand{\assgncom}[2]{#1\ :=\ #2}
\newcommand{\commitcom}[1]{\texttt{commit}\ #1}
\newcommand{\hypwrld}[1]{\texttt{hyp}\ \{#1\}}
\newcommand{\bigcdot}{\raisebox{-0.75ex}{\scalebox{2}{$\cdot$}}}
\newcommand{\partialfunc}{\rightharpoonup}
\newcommand{\config}[1]{\langle #1 \rangle}
\newcommand{\true}{\texttt{true}}
\newcommand{\false}{\texttt{false}}
\newcommand{\onestep}[1]{\rightarrow_{#1}}
\newcommand{\bigstep}[1]{\rightarrow_{#1}}
\newcommand{\com}{C}
\newcommand{\bool}{B}
\newcommand{\imp}{\textsc{Imp}\xspace}
\newcommand{\opal}{\textsc{Opal}\xspace}

\newcommand{\TLabel}[1]{\RightLabel{\textsc{T-#1}}}
\newcommand{\ELabel}[1]{\RightLabel{\textsc{E-#1}}}

% thanks to https://tex.stackexchange.com/a/132790
\def\checkmark{\tikz\fill[scale=0.4](0,.35) -- (.25,0) -- (1,.7) -- (.25,.15) -- cycle;}

\begin{document}

\maketitle

\section{Description}

Tiny turing-incomplete language based on \imp with sexps:

\section{Grammar}

\begin{minipage}{0.5\textwidth}
\begin{grammar}
<Com> ::= \skipcom
\alt \seqcom{\synt{Com}}{\synt{Com}}
\alt \synt{Node}.\synt{Var} := \synt{Sexp}
\alt \ifcom{\synt{Bool}}{\synt{Com}}{\synt{Com}}
\alt \withcom{\synt{Node}}{\synt{Com}}
\alt \atcom{\synt{Node}}{\synt{Com}}
\alt \assgncom{\synt{WorldVar}}{\hypwrld{c}}
\alt \commitcom{\synt{WorldVar}}

\end{grammar}
\end{minipage}
\begin{minipage}{0.5\textwidth}
\begin{grammar}

<Sexp> ::= $\varnothing$
\alt <Node>.<Var>
\alt <World>.<Node>.<Var>
\alt ( <Sexp> . <Sexp> )

<Bool> ::= <Sexp> = <Sexp>
\alt <Sexp> $\in$ <Sexp>
\alt <Bool> $\land$ <Bool>
\alt <Bool> $\lor$ <Bool>
\alt \true
\alt \false

\end{grammar}
\end{minipage}

\subsection{Values}

\begin{minipage}{0.5\textwidth}
\begin{grammar}

<SexpValue> ::= $\varnothing$
\alt ( <SexpValue> . <SexpValue> )

\end{grammar}
\end{minipage}
\begin{minipage}{0.5\textwidth}
\begin{grammar}

<BoolValue> ::= \true
\alt \false

\end{grammar}
\end{minipage}

\section{Typing Judgement}

Well-formedness conditions:
\begin{itemize}
\item All node accesses are permissioned via a \texttt{with} block
\item All variable accesses are defined (if they access a world, that world is defined and not committed)
\item All committed worlds are defined
\item Worlds are committed at most once
\end{itemize}

\subsection{Terminology}
\begin{tabular}{l c l}
$\Sigma$ : & $2^{\synt{Node} \times \synt{Var}}$ & Set of variables in scope \\
$\Omega$ : & $2^{\synt{World}}$ & Set of worlds in scope \\
$\Pi$ : & $2^{\synt{Node}}$ & Set of nodes giving permission to access variables
\end{tabular}

\subsection{Commands}

Commands follow an affine typing scheme for worlds, wherein worlds can be used \textit{at most} once. This is achieved by a conservative judgement over commands: declaring a hypothetical world puts it into scope, committing it takes it out of scope, and if statements take the intersection of available worlds after each branch.

The typing judgement for commands is a relation $R \subseteq \Sigma \times \Omega \times \Pi \times \synt{Com} \times \Omega$, where intuitively the first three members are the state before the command is executed, and the final $\Omega$ is the conservative judgement of the set of available worlds after the command is executed.

\begin{prooftree}
\AxiomC{}
\TLabel{Skip}
\UnaryInfC{$\Sigma, \Omega, \Pi \vdash \skipcom : \Omega$}
\end{prooftree}

\begin{prooftree}
\AxiomC{$\Sigma, \Omega, \Pi \vdash c_1 : \Omega'$}
\AxiomC{$\Sigma, \Omega', \Pi \vdash c_2 : \Omega''$}
\TLabel{Seq}
\BinaryInfC{$\Sigma, \Omega, \Pi \vdash c_1 ; c_2 : \Omega''$}
\end{prooftree}

\begin{prooftree}
\AxiomC{$\Sigma, \Omega, \Pi \vdash b : \textsc{Bool}$}
\AxiomC{$\Sigma, \Omega, \Pi \vdash c_1 : \Omega'$}
\AxiomC{$\Sigma, \Omega, \Pi \vdash c_2 : \Omega''$}
\TLabel{If}
\TrinaryInfC{$\Sigma, \Omega, \Pi \vdash \ifcom{b}{c_1}{c_2} : \Omega' \cap \Omega''$}
\end{prooftree}

\begin{prooftree}
\AxiomC{$\Sigma, \varnothing, \Pi \vdash c : \Omega'$}
\TLabel{AssignWorld}
\UnaryInfC{$\Sigma, \Omega, \Pi \vdash \assgncom{u}{\hypwrld{c}} : \Omega \cup \{u\}$}
\end{prooftree}

\begin{prooftree}
\AxiomC{$u \in \Omega$}
\TLabel{CommitWorld}
\UnaryInfC{$\Sigma, \Omega, \Pi \vdash \commitcom{u} : \Omega \setminus \{u\}$}
\end{prooftree}

\begin{prooftree}
\AxiomC{$\Sigma, \Omega, \Pi \cup \{n\} \vdash c : \Omega'$}
\TLabel{With}
\UnaryInfC{$\Sigma, \Omega, \Pi \vdash \withcom{n}{c} : \Omega'$}
\end{prooftree}

\begin{prooftree}
\AxiomC{$\Sigma, \Omega, \Pi \vdash c : \Omega'$}
\TLabel{At}
\UnaryInfC{$\Sigma, \Omega, \Pi \vdash \atcom{n}{c} : \Omega'$}
\end{prooftree}

\subsection{Booleans}

\begin{prooftree}
\AxiomC{}
\TLabel{True}
\UnaryInfC{$\Sigma, \Omega, \Pi \vdash \true : \textsc{Bool}$}
\end{prooftree}

\begin{prooftree}
\AxiomC{}
\TLabel{False}
\UnaryInfC{$\Sigma, \Omega, \Pi \vdash \false : \textsc{Bool}$}
\end{prooftree}

\begin{prooftree}
\AxiomC{$\Sigma, \Omega, \Pi \vdash b_1 : \textsc{Bool}$}
\AxiomC{$\Sigma, \Omega, \Pi \vdash b_2 : \textsc{Bool}$}
\TLabel{Conj}
\BinaryInfC{$\Sigma, \Omega, \Pi \vdash b_1 \land b_2 : \textsc{Bool}$}
\end{prooftree}

\begin{prooftree}
\AxiomC{$\Sigma, \Omega, \Pi \vdash b_1 : \textsc{Bool}$}
\AxiomC{$\Sigma, \Omega, \Pi \vdash b_2 : \textsc{Bool}$}
\TLabel{Disj}
\BinaryInfC{$\Sigma, \Omega, \Pi \vdash b_1 \lor b_2 : \textsc{Bool}$}
\end{prooftree}

\begin{prooftree}
\AxiomC{$\Sigma, \Omega, \Pi \vdash s_1 : \textsc{Sexp}$}
\AxiomC{$\Sigma, \Omega, \Pi \vdash s_2 : \textsc{Sexp}$}
\TLabel{Eq}
\BinaryInfC{$\Sigma, \Omega, \Pi \vdash s_1 = s_2 : \textsc{Bool}$}
\end{prooftree}

\begin{prooftree}
\AxiomC{$\Sigma, \Omega, \Pi \vdash s_1 : \textsc{Sexp}$}
\AxiomC{$\Sigma, \Omega, \Pi \vdash s_2 : \textsc{Sexp}$}
\TLabel{Mem}
\BinaryInfC{$\Sigma, \Omega, \Pi \vdash s_1 \in s_2 : \textsc{Bool}$}
\end{prooftree}

\subsection{Sexps}

\begin{prooftree}
\AxiomC{}
\TLabel{EmptySexp}
\UnaryInfC{$\Sigma, \Omega, \Pi \vdash \varnothing : \textsc{Sexp}$}
\end{prooftree}

\begin{prooftree}
\AxiomC{$n \in \Pi$}
\AxiomC{$(n, v) \in \Sigma$}
\TLabel{Variable}
\BinaryInfC{$\Sigma, \Omega, \Pi \vdash n.v : \textsc{Sexp}$}
\end{prooftree}

\begin{prooftree}
\AxiomC{$u \in \Omega$}
\AxiomC{$n \in \Pi$}
\AxiomC{$(n, v) \in \Sigma$}
\TLabel{Weight}
\TrinaryInfC{$\Sigma, \Omega, \Pi \vdash u.n.v : \textsc{Sexp}$}
\end{prooftree}

\begin{prooftree}
\AxiomC{$\Sigma, Omega, \Pi \vdash s_1 : \textsc{Sexp}$}
\AxiomC{$\Sigma, Omega, \Pi \vdash s_2 : \textsc{Sexp}$}
\TLabel{Cons}
\BinaryInfC{$\Sigma, Omega, \Pi \vdash (s_1 . s_2) : \textsc{Sexp}$}
\end{prooftree}


\section{Big Step Semantics}
\subsection{Terminology}

\begin{tabular}{l c p{10cm}}
$\sigma$ : & $\synt{Node} \times \synt{Var} \partialfunc \synt{SexpValue}$ & Partial function representing each node's store \\\\
$\omega$ : & $\synt{World} \partialfunc \sigma$ & Partial function representing each executed world's final store \\\\
$\pi$ : & $2^{\synt{Node}}$ & Current authorized set of principals \\\\
$\rho$ : & $\synt{Node}$ & Current execution location \\\\
$\mu$ : & $(\synt{Sexp} \times \synt{Sexp}) \partialfunc \synt{Sexp}$ & Partial(!) function which merges two divergent data structures \\\\
$\Downarrow_{\synt{Com}}$ : & $(\synt{Com} \times \sigma \times \omega \times \pi \times \rho \times \mu) \partialfunc (\sigma \times \omega)$  & Commands step to a new store and new set of hypothetical worlds\\\\
$\Downarrow_{\synt{Sexp}}$ : & $(\synt{Sexp} \times \sigma \times \pi) \partialfunc \synt{SexpValue}$ & Sexps step to a value if they can be evaluated \\\\
$\Downarrow_{\synt{Bool}}$ : & $(\synt{Bool} \times \sigma \times \pi) \partialfunc \synt{BoolValue}$ & Bools step to a value if they can be evaluated \\\\
\end{tabular}

\subsection{Basic commands}
\begin{prooftree}
\AxiomC{}
\ELabel{Skip}
\UnaryInfC{$\config{\skipcom, \sigma, \omega, \pi, \rho, \mu} \Downarrow \config{\sigma, \omega}$}
\end{prooftree}

\begin{prooftree}
\AxiomC{$\config{c_1, \sigma, \omega, \pi, \rho, \mu} \Downarrow \config{\sigma', \omega'}$}
\AxiomC{$\config{c_2, \sigma', \omega', \pi, \rho, \mu} \Downarrow \config{\sigma'', \omega''}$}
\ELabel{Seq}
\BinaryInfC{$\config{\seqcom{c_1}{c_2}, \sigma, \omega, \pi, \rho, \mu} \Downarrow \config{\sigma'', \omega''}$}
\end{prooftree}

\begin{prooftree}
\AxiomC{$\config{b, \sigma, \pi} \Downarrow \true$}
\AxiomC{$\config{c_1, \sigma, \omega, \pi, \rho, \mu} \Downarrow \config{\sigma', \omega'}$}
\ELabel{If-True}
\BinaryInfC{$\config{\ifcom{b}{c_1}{c_2}, \sigma, \omega, \pi, \rho, \mu} \Downarrow \config{\sigma', \omega'}$}
\end{prooftree}

\begin{prooftree}
\AxiomC{$\config{b, \sigma, \pi} \Downarrow \false$}
\AxiomC{$\config{c_2, \sigma, \omega, \pi, \rho, \mu} \Downarrow \config{\sigma', \omega'}$}
\ELabel{If-False}
\BinaryInfC{$\config{\ifcom{b}{c_1}{c_2}, \sigma, \omega, \pi, \rho, \mu} \Downarrow \config{\sigma', \omega'}$}
\end{prooftree}

\subsection{Distribution commands}
In this semantics, \textsc{At} is effectively a noop/passthrough, resulting in behavior that would be identical with or without it (modulo endorsement taking location into account).

\textsc{With} is also a passthrough in a similar regard: its runtime effects consist of asking the specified user for permission to run the specified command on the current machine (represented by $\config{n, \rho, c}\ \checkmark$ in the semantics -- the details of this function are implementation dependent).

\begin{prooftree}
\AxiomC{$\config{c, \sigma, \omega, \pi, n, \mu} \Downarrow \config{\sigma', \omega'}$}
\ELabel{At}
\UnaryInfC{$\config{\atcom{n}{c}, \sigma, \omega, \pi, \rho, \mu} \Downarrow \config{\sigma', \omega'}$}
\end{prooftree}

\begin{prooftree}
\AxiomC{$\config{n, \rho, c}\ \checkmark$}
\AxiomC{$\config{c, \sigma, \omega, \pi \cup \{n\}, \rho, \mu} \Downarrow \config{\sigma', \omega'}$}
\ELabel{With}
\BinaryInfC{$\config{\withcom{n}{c}, \sigma, \omega, \pi, \rho, \mu} \Downarrow \config{\sigma', \omega'}$}
\end{prooftree}

\subsection{Hypothetical commands}

\begin{prooftree}
\AxiomC{$\config{c, \sigma, \varnothing, \pi, \rho, \mu} \Downarrow \config{\sigma_{\text{hyp}}, \omega_{\text{hyp}}}$}
\ELabel{Hyp}
\UnaryInfC{$\config{\assgncom{u}{\hypwrld{c}}, \sigma, \omega, \pi, \rho, \mu} \Downarrow \config{\sigma, \omega[u \mapsto \sigma_{\text{hyp}}]}$}
\end{prooftree}

\begin{prooftree}
\AxiomC{$\omega[u] = \sigma_{\text{hyp}}$}
\AxiomC{$\forall v \in \sigma_{\text{hyp}}.\ \sigma_{\text{merge}}[v] = \mu(\sigma_\text{curr}[v], \sigma_\text{hyp}[v])$}
\AxiomC{$\forall v \notin \sigma_{\text{hyp}}. \nexists s. \sigma_{\text{merge}}[v] = s$}
\ELabel{Commit}
\TrinaryInfC{$\config{\commitcom{u}, \sigma_{\text{curr}}, \omega, \pi, \rho, \mu} \Downarrow \config{\sigma_{\text{merge}} ; \sigma_{\text{curr}}, \omega}$}
\end{prooftree}

\subsection{Sexps}

\begin{prooftree}
\AxiomC{}
\ELabel{EmptySexp}
\UnaryInfC{$\config{\varnothing, \sigma, \omega, \pi} \Downarrow \varnothing$}
\end{prooftree}

\begin{prooftree}
\AxiomC{$\config{s_1, \sigma, \omega, \pi} \Downarrow s_1'$}
\AxiomC{$\config{s_2, \sigma, \omega, \pi} \Downarrow s_2'$}
\ELabel{Cons}
\BinaryInfC{$\config{(s_1 . s_2), \sigma, \omega, \pi} \Downarrow (s_1' . s_2')$}
\end{prooftree}

\begin{prooftree}
\AxiomC{$n \in \pi$}
\AxiomC{$\sigma[n,v] = s$}
\ELabel{Var}
\BinaryInfC{$\config{n.v, \sigma, \omega, \pi} \Downarrow s$}
\end{prooftree}

\begin{prooftree}
\AxiomC{$n \in \pi$}
\AxiomC{$\omega[u] = \sigma_\text{hyp}$}
\AxiomC{$\sigma_\text{hyp}[n,v] = s$}
\ELabel{Weight}
\TrinaryInfC{$\config{u.n.v, \sigma, \omega, \pi} \Downarrow s$}
\end{prooftree}

\subsection{Bools}
\begin{prooftree}
\AxiomC{}
\ELabel{True}
\UnaryInfC{$\config{\true, \sigma, \omega, \pi} \Downarrow \true$}
\end{prooftree}
\begin{prooftree}
\AxiomC{}
\ELabel{False}
\UnaryInfC{$\config{\false, \sigma, \omega, \pi} \Downarrow \false$}
\end{prooftree}

\begin{prooftree}
\AxiomC{$\config{b_1, \sigma, \omega, \pi} \Downarrow \true$}
\AxiomC{$\config{b_2, \sigma, \omega, \pi} \Downarrow \true$}
\ELabel{AndTrue}
\BinaryInfC{$\config{b_1 \land b_2, \sigma, \omega, \pi} \Downarrow \true$}
\end{prooftree}

\begin{prooftree}
\AxiomC{$\config{b_1, \sigma, \omega, \pi} \Downarrow \false$}
\ELabel{AndFalseL}
\UnaryInfC{$\config{b_1 \land b_2, \sigma, \omega, \pi} \Downarrow \false$}
\end{prooftree}

\begin{prooftree}
\AxiomC{$\config{b_2, \sigma, \omega, \pi} \Downarrow \false$}
\ELabel{AndFalseR}
\UnaryInfC{$\config{b_1 \land b_2, \sigma, \omega, \pi} \Downarrow \false$}
\end{prooftree}

\begin{prooftree}
\AxiomC{$\config{b_1, \sigma, \omega, \pi} \Downarrow \false$}
\AxiomC{$\config{b_2, \sigma, \omega, \pi} \Downarrow \false$}
\ELabel{OrFalse}
\BinaryInfC{$\config{b_1 \lor b_2, \sigma, \omega, \pi} \Downarrow \false$}
\end{prooftree}

\begin{prooftree}
\AxiomC{$\config{b_1, \sigma, \omega, \pi} \Downarrow \true$}
\ELabel{OrTrueL}
\UnaryInfC{$\config{b_1 \lor b_2, \sigma, \omega, \pi} \Downarrow \true$}
\end{prooftree}

\begin{prooftree}
\AxiomC{$\config{b_2, \sigma, \omega, \pi} \Downarrow \true$}
\ELabel{OrTrueR}
\UnaryInfC{$\config{b_1 \lor b_2, \sigma, \omega, \pi} \Downarrow \true$}
\end{prooftree}

\begin{prooftree}
\AxiomC{$\config{s_1, \sigma, \omega, \pi} \Downarrow \varnothing$}
\AxiomC{$\config{s_2, \sigma, \omega, \pi} \Downarrow \varnothing$}
\ELabel{EqTrue}
\BinaryInfC{$\config{s_1 = s_2, \sigma, \omega, \pi} \Downarrow \true$}
\end{prooftree}

\begin{prooftree}
\AxiomC{$\config{s_1, \sigma, \omega, \pi} \Downarrow ({{s_v}_1}_1 . {{s_v}_1}_2)$}
\AxiomC{$\config{s_2, \sigma, \omega, \pi} \Downarrow \varnothing$}
\ELabel{EqFalseL}
\BinaryInfC{$\config{s_1 = s_2, \sigma, \omega, \pi} \Downarrow \false$}
\end{prooftree}

\begin{prooftree}
\AxiomC{$\config{s_1, \sigma, \omega, \pi} \Downarrow \varnothing$}
\AxiomC{$\config{s_2, \sigma, \omega, \pi} \Downarrow ({{s_v}_2}_1 . {{s_v}_2}_2)$}
\ELabel{EqFalseR}
\BinaryInfC{$\config{s_1 = s_2, \sigma, \omega, \pi} \Downarrow \false$}
\end{prooftree}

\begin{prooftree}
\AxiomC{$\config{s_1, \sigma, \omega, \pi} \Downarrow ({{s_v}_1}_1 . {{s_v}_1}_2)$}
\AxiomC{$\config{s_2, \sigma, \omega, \pi} \Downarrow ({{s_v}_2}_1 . {{s_v}_2}_2)$}
\AxiomC{$\config{{{s_v}_1}_1 = {{s_v}_2}_1 \land {{s_v}_1}_2 = {{s_v}_2}_2, \sigma, \pi} \Downarrow b$}
\ELabel{EqProp}
\TrinaryInfC{$\config{s_1 = s_2, \sigma, \omega, \pi} \Downarrow b$}
\end{prooftree}

\begin{prooftree}
\AxiomC{$\config{s_2, \sigma, \omega, \pi} \Downarrow \varnothing$}
\ELabel{MemFalse}
\UnaryInfC{$\config{s_1 \in s_2, \sigma, \omega, \pi} \Downarrow \false$}
\end{prooftree}

\begin{prooftree}
\AxiomC{$\config{s_2, \sigma, \omega, \pi} \Downarrow ({{s_v}_2}_1 . {{s_v}_2}_2)$}
\AxiomC{$\config{s_1 = {{s_v}_2}_1 \lor s_1 \in {{s_v}_2}_2, \sigma, \omega, \pi} \Downarrow b$}
\ELabel{MemProp}
\BinaryInfC{$\config{s_1 \in s_2, \sigma, \omega, \pi} \Downarrow b$}
\end{prooftree}

\textsc{EqProp} and \textsc{MemProp} are well-founded proof trees since the sexp step is idempotent, and the two props decrease on size of sexp, so the maximum depth of the proof tree is proportional to the maximum depth of the sexp values

\end{document}


%%% Local Variables:
%%% mode: latex
%%% TeX-master: t
%%% End:
